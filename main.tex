%...%

%Nel root del file .tex deve esserci la cartella miaclasse.cls e
%il file pacchetti.tex nella cartella Sottofile/, poi è una buona idea avere una cartella apposta per le immagini e
%un altro file .tex in cui scrivere il corpo del testo (anche se non è necessario)
%nel corpo del testo non deve esserci nulla più del testo

\documentclass{miaclasse}

%...%

%se mi serve la funzione import
\usepackage{import}

%Percorso immagini
%\graphicspath{{Immagini/}}

%...%

%Ho fatto un file con tutti i package usati per rendere più snello il file
\import{Sottofile/}{pacchetti}


%...%

%...%

%Inizio documento

\begin{document}


%Roba per il titolo
\titolo{Appunti di Controllo dei Sistemi Meccanici}
\autore{Bozza, non condividere}
%comando della miaclasse per il titolo
\primapag

\pagestyle{fancy}


%...%



%se voglio posso mettere l'indice togliendo le percentuali
%\renewcommand*\contentsname{Indice}
%\tableofcontents

%...%

\import{Sottofile/}{Introduzione}
\import{Sottofile/}{Modellistica}
\import{Sottofile/}{DimensionamentoSceltaRiduttori}
\import{Sottofile/}{DimensionamentoSceltaViti}

\appendix
\import{Sottofile/}{Appendice_controlli}
\import{Sottofile/}{Appendice_MV}

\end{document}
