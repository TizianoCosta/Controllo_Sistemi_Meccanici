%...%

%Nel root del file .tex deve esserci la cartella miaclasse.cls e
%il file pacchetti.tex nella cartella Sottofile/, poi è una buona idea avere una cartella apposta per le immagini e
%un altro file .tex in cui scrivere il corpo del testo (anche se non è necessario)
%nel corpo del testo non deve esserci nulla più del testo

\documentclass{miaclasse}

%...%

%se mi serve la funzione import
\usepackage{import}

%Percorso immagini
%\graphicspath{{Immagini/}}

%...%

%Ho fatto un file con tutti i package usati per rendere più snello il file
\import{Sottofile/}{pacchetti}


%...%

%...%

%Inizio documento

\begin{document}


%Roba per il titolo
\titolo{Appunti Esempio}
\autore{Tiziano Costa}
%comando della miaclasse per il titolo
\primapag

\pagestyle{fancy}


%...%



%se voglio posso mettere l'indice togliendo le percentuali
\renewcommand*\contentsname{Indice}
\tableofcontents



%Posso scrivere roba qui

 
%...%


% o in alternativa posso scrivere roba su un altro file libero 
% \input{appunti_esempio}
%oppure se ho una cartella con vari file uso import (richiede il pkg import)
\import{Sottofile/}{file1}

\chapter{Fuori da sottofile}
Devo scrivere qui?

\section{Fuori da sottofile}
Tanto per vedere se va bene, numero sezione: \thesection
\section{Fuori da sottofile2}
Tanto per vedere se va bene, numero sezione: \thesection


\end{document}
