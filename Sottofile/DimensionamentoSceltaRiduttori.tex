\capitolo{Dimensionamento e Scelta dei Riduttori}
Con dimensionamento di riduttori si intende la valutazione di aspetti strutturali. La scelta dei riduttori invece coinvolge valutazioni differenti quali dinamiche, di controllo e come si relazionano.

\paragrafo{Come mai i riduttori?}
La scelta di utilizzare riduttori è legata a diversi aspetti:
\begin{enumerate}
    \item Adattamento del campo operativo di motore e carico. Solitamente i motori hanno coppie minori di quelle che sono richieste al carico, mentre possono fornire velocità molto superiori quelle di operazione, in questi casi il riduttore permette di ridurre la velocità e di ottenere una maggior coppia.
    \item Ottimizzazione del sistema. Un riduttore opportunamento scelto permette di ridurre costi, ingombro, consumo energetico e facilitare il controllo
    \item Facilita il controllo. Un riduttore ben scelto permette di linearizzare la dinamica del sistema. Per esempio avendo un carico di momento di inerzia variabile con la posizione, l'uso del riduttore permette di vedere dal motore una $J_{eq}=J_m+J_c\tau_r^2\simeq J_m$, quindi un momento di inerzia minore e circa costante.  
\end{enumerate}

\begin{figure}[h]
    \centering
    \includegraphics[width=0.2\textwidth]{Immagini/campo_operativo_riduttore.png}
    \includegraphics[width=0.18\textwidth,angle=-5]{Immagini/riduttore_epicicloidale.png}
\end{figure}

\paragrafo{Riduttori Epicicloidali:}
Il riduttore più impiegato nell'industria è quello epicicloidale, questa tipologia è caratterizzata da: buona riduzione; compattezza in termini di rapporto coppia volume; buona distribuzione delle forze; non è invertente; alberi di ingresso e uscita sono coassiali (permettendo una maggior facilità di layout).

\paragrafo{Parametri Progettuali:}
I parametri di progetto di un riduttore noti sono quelli del carico ($J_c$, $C_c$) e legge oraria ($\theta_c(t)$, $\VelAng_c(t)$, $\AccAng_c(t)$); quelli ignoti sono quelli di motore ($J_m$, $C_m$) e riduttore ($\tau_r$, $J_r$, $\eta_r$).
Perciò nel caso di motore riduttore, carico una formulazione più utile, e solitamente usata per determinare la coppia del motore è: 
\[C_m = \left(J_m+J_r\right)\frac{\AccAng_c}{\tau_r} + (C_c + J_c \AccAng_c)\frac{\tau_r}{\eta_r}\]
Il momento di inerzia del riduttore è unione di diversi contributi, ed è fondamentale capire quale sia l'albero a cui è riportato (di solito quello veloce).

\sezione{Step di Dimensionamento}
Per il dimensionamento del riduttore la prima cosa da fare è determinare la taglia, cioè la coppia erogabile $C_r$.
Dipende da considerazioni "strutturali" di tipo normativo\footnote{Non nel senso di legalmente vincolante, più come modo tipico di fare il dimensionamento.} e valutazioni di tipo dinamico (vibrazioni).

\sottosezione{Normativa:}
Il dimensionamento è basato su coppia e velocità, verificando in base al duty cycle la potenza trasmessa dal riduttore per un dato ciclo di riferimento ripetitivo.
In particolare in base al duty cycle vi sono due categorie: 
\begin{itemize}
    \item Regime Intermittente per $<60\%$, per cui occorrono valutazioni su coppia e velocità massime; 
    \item Regime Continuativo per $>60\%$, per cui occorrono valutazioni su coppia e velocità massime e medie.
\end{itemize}
In particolare le valutazioni per coppie e velocità massime andranno effettuate per punti di maggior criticità, ossia la coppia all'albero lento, la velocità all'albero veloce.

\sottosottosezione{Coppia albero lento}
All'albero lento ho la coppia maggiore, quindi è il punto critico. 
Definisco $T_2$ coppia (torque) dell'albero lento (2) la risultante delle coppie applicate sull'albero lento (es coppia e coppia dell'inerzia di carico).
Di $T_2$ è tutto noto, perciò si può procedere subito a questo conto, il cui valore risultante va confrontato con il valore di massima coppia erogata all'albero lento: 
\[1.2\cdot \max{\abs{T_2(t)}}\cdot f_s < T_{2,B}\]
dove:
\begin{itemize}
    \item $1.2$ è un fattore di sicurezza legato alle incertezze di conoscenza dei parametri, valori tipici possono essere $[20\%\div 35\%]$
    \item $f_s$ è un fattore di servizio, valore a catalogo che viene interpretato in modo diverso da ciascun costruttore
    \item $T_{2,B}$ è la coppia massima, è un valore a catalogo e caratterizza la taglia del riduttore. La "B" sta per Bound, in italiano Limite.
\end{itemize}

\paragrafo{Coppia Media:}
Per determinare la coppia media si fa una media cubica ponderata lo spazio di modo che rappresenti la fatica a ciclo variabile, gli effetti termici, calcolata come:
\[
T_{2,media} = \sqrt[3]{\frac{\int^{T_{ciclo}}_0 \abs{T_2^3(t)\VelAng_c(t)dt}}{\int^{T_{ciclo}}_0 \abs{\VelAng_c(t)}dt}}
\]
Anche per questa si va ad applicare un fattore di sicurezza rispetto il punto di funzionamento ottimale $1.2\cdot T_{2,media}<T_{2,nom}$.

\paragrafo{Catalogo:}
Da catalogo COBRA è chiaro come indipendentemente dal numero di stadi la coppia rimanga simile, ed è proprio perché è quella che definisce la taglia. 
Inoltre la maggior parte dei parametri sono simili all'interno di un unica taglia.
Per forti riduzioni (1/10 e 1/100) si nota come le coppie $T_{2,B}$ e $T_{2,N}$ siano differenti tra loro, questo perché la geometria delle ruote saranno molto diverse portando ad avere un ingranamento qualitativamente peggiore.

\sottoparagrafo{Coppia di Arresto:}
La coppia di arresto $T_{2,not}$ è la massima coppia che il riduttore può sopportare in caso di arresto di emergenza (circa 100 volte in vita)\footnote{NB: La coppia di arresto non è quella da usare per la frenata all'interno del ciclo operativo!}.

\sottosottosezione{Velocità all'albero veloce}
Il limite di velocità massima è legato a problemi di sbilanciamento, ruotando crea una forzante armonica pari alla velocità di rotazione e cresce con velocità e sbilanciamento. 
Deve valere $\max{\VelAng_m(t)}< n_{1,max}$ massima velocità ammissibile, cui aggiungere un fattore di sicurezza.
Diventa problematica non appena $\VelAng \simeq \omega_{N, flessionale}$.
Un aspetto da non sottovalutare è la presenza di eventuali sovraelongazioni nella forma d'onda di velocità legate a controllo reale.
Aumentare la taglia aumenta la massa e la dimensione, perciò aumenta lo sbilanciamento e quindi cala $n_{1,max}$.

\paragrafo{Velocità media all'albero veloce:}
La velocità media deve essere limitata per limitare fenomeni termici e $T_{2,media}$. La velocità media è data dallo spazio totale percorso (in assoluto) su tempo ciclo totale, questo permette di considerare i momenti di arresto marcia in cui il riduttore va a raffreddarsi. La velocità media dev'essere inferiore al valore a catalogo $v_{media} < n_{1,N}$. Anche in questo caso si utilizza un coefficiente di sicurezza del $20\%$. Alcuni costruttori tuttavia usano convenzioni differenti per determinare la velocità media, prestare attenzione alla definizione a catalogo.
Aumentare la taglia aumenta la massa, quindi l'attrito, ossia il calore generato perciò cala anche la $n_{1,N}$, ma in modo poco marcato.

\sottosottosezione{Rapporto di riduzione}
Le velocità sono critiche all'albero veloce, tuttavia $\VelAng_m=\frac{\VelAng_c}{\tau_r}$ di cui a noi è nota solo la velocità del carico. Questo è coerente con il tipo di dimensionamento voluto, perché lo studio di $\tau_r$ si associa ad ulteriori ragionamenti, il vincolo legato a quell'espressione sarà usato per fare una ottimizzazione vincolata.

\paragrafo{Velocità massima all'albero veloce:}
$n_{1,max}>\max{\VelAng_m(t)}$, che in termini di rapporto di velocità diventa $\tau_r>\frac{\max{\VelAng_c}}{n_{1,max}}$.

\paragrafo{Velocità media all'albero veloce:}
$n_{1,N}>\VelAng_m^{media}$, che in termini di rapporto di velocità diventa $\tau_r > \frac{\VelAng_c^{media}}{n_{1,N}}$.

\paragrafo{Velocità limite:}
Tuttavia anche in queste ultime disequazioni è nota la velocità di carico, per le velocità limite bisogna considerare di aver già determinato la coppia, quindi la taglia del riduttore, da cui posso determinare dei valori di velocità limite che saranno abbastanza simili a quelli poi usati effettivamente.

\sottoparagrafo{Se la taglia di quel costruttore non ha valori che vadano bene?}
Se la taglia del costruttore disponibile ha valori di velocità che non sono compatibili con quelli richiesti: per la velocità media c'è poco da fare, nel senso che è determinata dalla legge oraria, cioè dal tipo di funzionalità richiesta e variarla significa non dare al cliente ciò che ha richiesto; per la velocità massima invece c'è più margine di scelta perché il coefficiente di sicurezza è maggiore.

\sottosezione{Valutazioni su Riduttore}
Ci sono diversi aspetti strutturali da considerare per la scelta di un riduttore.

\sottosottosezione{Dimensione dei vari stadi in un multistadio:}
In un riduttore multistadio le dimensioni dei singoli stadi non sono le stesse, perché le coppie che devono sopportare sono differenti, perciò non è esattamente come considerare due singolo stadio.

% inserire immagine rappresentativa

\sottosottosezione{Massa}
La massa del riduttore è proporzionale alla taglia e al numero di stadi. 
In generale maggiore massa comporta un maggior costo, nonostante ciò l'uso di multistadio può portare un calo del costo del motore e quindi dell'energia (vedi lezione su $\tau$ ottimo).

\sottosottosezione{Momento di inerzia dell'albero veloce}
All'interno di una stessa taglia i valori di inerzia sono molto simili perché le ruote hanno modulo di denti e sezioni sugli alberi simili se non identici, quindi la massa nonché l'inerzia saranno circa uguali.
Le differenze sono legate alle dimensioni delle ruote quindi satelliti e sole, dove il sole porta ad avere un momento di inerzia leggermente maggiore, quindi $J_{1/10}<J_{1/3}$.
Tra le varie taglie invece il momento di inerzia aumenta in modo è più che lineare, quasi quadratico.
A parità di taglia, per diverso numero di stadi, nonostante la massa aumenti nel multistadio, il momento di inerzia cala, perché va considerata l'inerzia equivalente all'albero veloce, quindi: $J_r=J_{1st}+J_{2st}\tau_{1st}^2$.

%inserire immagine con grafico quasi quadratico e momento di inerzia equivalente

\sottosottosezione{Rendimento}
Il rendimento varia con l'aumentare del numero di stadi, perché $\eta_r=\Pi_i \eta_i$, valori tipici sono:
\begin{itemize}
    \item 1 stadio: $96\% \div 98\%$
    \item 2 stadio: $94\% \div 95\%$
    \item 3 stadio: $90\% \div 92\%$
\end{itemize}

\sottosottosezione{Rigidezza}
In condizioni ideali bloccare l'albero veloce comporta il blocco dell'albero lento, tuttavia in condizioni reali presenterà una variazione circa lineare di tipo elastico (soprattutto nella zona di lavoro), a cui si va ad aggiungere una componente non lineare legata al gioco.
Perciò può essere introdotto un nuovo modello per il riduttore che consideri la rigidezza reale considerandolo come una molla.
Il sistema passa ad avere due gradi di libertà, non è più valida l'espressione $\VelAng_m \neq \frac{\VelAng_c}{\tau_r}$, ma diventerà, considerando $T(s)$ funzione di trasferimento:
\[\theta_c(s)=s \theta_m \tau_r + T(s)\]
Nel caso multistadio la rigidezza da considerare è una rigidezza equivalente data dalla serie, perciò sarà minore delle rigidezze dei singoli stadi: $K_r=\frac{1}{\frac{\tau_2^2}{k_1}+\frac{1}{k_2}}<k_1,k_2$.

% inserire immagine modello nuovo e caso multistadio

\paragrafo{Cataloghi:}
Nei cataloghi la rigidezza è definita all'albero lento, perché permette di fare calcoli più semplice come deformazione vista dal carico.
Nel caso del costruttore visto a lezione $k=R_d$, si nota come aumenti con la taglia in modo circa lineare, perché aumentare la sezione e massa del riduttore aumenta la resistenza torsionale. \footnote{Esistono riduttori epicicloidali con rigidezze particolarmente elevate sfruttanto una rivisitazione dei riduttori epicicloidali classici, vedi: Galaxie della Wittenstein e Cyclodrive della Sumimoto.}