\capitolo{Dimensionamento e Scelta di Viti a Ricircolo di Sfere}
I riduttori vite madrevite classici sono a strisciamento e questa caratteristica li rende particolarmente soggetti all'attrito, quindi a ridotto rendimento e irreversibilità del moto. Queste sono caratteristiche desiderabili per strutture, meno per organi in moto, si fa largo la necessità di rendimenti superiori, quindi vengono utilizzate tecnologie differenti.

La principale delle tecnologie alternative è quella di riduttori a vite a ricircolo di sfere in cui lo spostamento viene trasmesso dalle sfere che rotolando permettono di avere ridotto attrito, quindi rendimento maggiore, indicativamente \( \eta_{R\rightarrow T} \simeq 0.97 \simeq \eta_{T\rightarrow R} \).

Le sfere sono poste all'interno di gole su cui possono scorrere, presenti al posto dei denti nella vite e madrevite, chiamata in questo contesto chiocciola.

\paragrafo{Configurazioni cinematiche:}
In termmini di cinematica le viti a ricircolo di sfere sono equivalenti alle viti-madreviti a strisciamento, in particolare vale ancora \( \tau_v = \frac{p}{2\pi} = \frac{\Delta x_{rel}}{\Delta \theta_{rel}} \), dove gli spostamenti sono considerati relativi.
In seguito sono rappresentate le varie combinazioni di movimenti assoluti.

\begin{figure}[h]
    \centering
    \includegraphics[width=0.9\textwidth]{Immagini/confi_din_vite_ricircolo.png}
    \caption{4 possibili configurazioni rotazione-traslazione relative}
\end{figure}

\sezione{Parametri di scelta}
La scelta di viti a ricircolo di sfere non è banale perché ha dipendenza da un alto numero di combinazioni possibili di diametri viti e sfere, passo della vite, vita utile, numero di principi e tipologia di materiali.

\paragrafo{Numero di principi:}
Con numero di principi si intendono i singoli percorsi indipendenti di ricircolo delle sfere. All'interno di un passo della vite potrebbe esserci sufficiente spazio per inserire un ulteriore ricircolo di sfere, così nasce una vite a due principi.

\sottosezione{Dinamica}
Similmente a come visto per i riduttori si vanno a definire la forza applicata sul carico: \( C_m = \left( J_m + J_v + M \frac{\tau_v^2}{\eta_v} \right) \AccAng_m + F \frac{\tau_v}{\eta_v} \), da cui, a seguito di opportuna manipolazione, si ottiene: \( C_m = \left(J_m + J_v\right) \frac{\Ddot x}{\tau_v} + \frac{\tau_v}{\eta_v} \left( M\Ddot x + F \right) \), dove:
\[ F_a = M\Ddot x + F \]
risultante delle forze assiali applicate alla chiocciola (equivalente a \( T_2 \) dei riduttori).

\begin{figure}[h]
    \centering
    \includegraphics[width=0.4\textwidth]{Immagini/mot_vite_ricircolo.png}
    \includegraphics[width=0.3\textwidth]{Immagini/principi_viti_ricircolo.png}
    \caption{Schema motore e vite a ricircolo di sfere sx; Uno vs Due principi dx}
\end{figure}

\sottosottosezione{Forza assiale media}
Anche in questo caso nel regime stazionario si fanno valutazioni sulla media, e in modo simile a \( T_{2,media} \) si ottiene:
\[ F_{a, media} = \sqrt[3]{\frac{\int^{T_{ciclo}}_0 \abs{F_a^3(t)\dot{x}_c(t)}dt}{\int^{T_{ciclo}}_0 \abs{\dot{x}_c(t)}dt}} \]
E anche in questo caso per moto ad andamento trapezoidale l'integrale diventerà una sommatoria.

\sottosottosezione{Capacità di Carico Dinamico}
Per capacità di carico dinamico \( C_D [N] \) si intende nei cataloghi la forza assiale applicata alla  chiocciola che garantisce con probabilità di sopravvivenza del \(90\%\) la vita utile di \(10^6\) giri relativi vite-chiocciola.

\paragrafo{Migliorare la probabilità:}
Per migliorare la probabilità occorre utilizzare un fattore di affidabilità \(<1\) che vada a penalizzare il valore a catalogo. Si tratta di valori ricavati statisticamente, non hanno significato di coefficiente di sicurezza.

\paragrafo{Utilizzo di CCD in verifica:}
Noti \(C_D, F_{a, media}\), volendo calcolare la vita utile della chiocciola per il \(90\%\) di probabilità di sopravvivenza si utilizza la formula di Wohler: \( C_D^3 10^6 = F_{a,media}^3 L_N \), con \(L_N\) vita utile in termini di numero di rotazioni.

\paragrafo{Utilizzo di CCD in dimensionamento:}
Noto \(F_{a,media}\), data dalla specifica in termini di giri sulla vita utile \(L_N^{des}\) desiderata, è possibile calcolare \(C_D\) idoneo a garantire la vita desiderata al \(x\%\) invertendo la formula e utilizzando un opportuno fattore di affidabilità.

\paragrafo{Vita utile in ore:}
Solitamente non è nota la vita utile in numero di giri relativi, è più comodo passare in ore di lavoro, quindi 
\[ L_H = L_N \frac{1}{\VelAng_v} \frac{1}{60} [h] = L_N \frac{p}{3600 \cdot \dot{x}_{media}} \]
che si traduce in una CCD richiesta: 
\[ C_{D,rich} \geqslant F_{a,media} \sqrt[3]{\frac{L_H^{des}\dot{x}^{des} \cdot 3600}{p \cdot 10^6}} \cdot f_s \]

\sottoparagrafo{Fattore di Shock/Servizio:}
Alla CCD in esame va moltiplicato un fattore di shock o servizio \( f_s \), i cui valori possono cambiare tra \( 1.2 \div 3 \), se possibile conviene utilizzare un modello elasto-dinamico per avere una idea delle forze in gioco e evitare di sovradimensionare la chiocciola.

\paragrafo{Dipendenze della CCD richiesta:}
La CCD è dipendente direttamente da velocità traslante e vita utile, mentre è inversamente dipendente dal passo \( p \downarrow, C_D \uparrow\). La dipendenza inversa del passo risulta chiara immaginando che i passi minori portano a un maggior numero di giri relativi a pari velocità.

\paragrafo{Dipendenze della CCD:}
Il carico dinamico sulla chiocciola dipende principalmente dal volume delle sfere e dal materiale delle stesse, quindi maggiore lo spazio disponibile alle sfere meglio è, perciò aumenta con il diametro delle viti \(d_v\), con l'utilizzo di sfere di raggio maggiore e l'utilizzo di viti a più principi.
Come regola generale conviene avere meno sfere più grandi che tenderanno a raggiungere un volume complessivo maggiore, inoltre garantiscono una maggior resistenza.

\begin{figure}[h]
    \centering
    \includegraphics[width=0.4\textwidth]{Immagini/CCD_chiocciola_sfere.png}
    \caption{Dipendenza volume sfere}
\end{figure}

\sottosezione{Velocità limite delle sfere}
Durante il moto le sfere ruotano tra le due gole di vite e chiocciola, ma in questo movimento vanno anche a strisciare tra loro. Per velocità elevate risulta un aumento di riscaldamento e un aumento usura legato alla formazione di particolato di sfere.
La relazione da verificare è \( v_{sfere} = \VelAng_v \frac{d}{2} < v_{lim} \) dove \(d\) è la distanza tra centri delle due sfere opposte, e con velocità limite specifica della chiocciola.

Nei cataloghi non si tiene conto della velocità della sfera, ma del doppio, e viene chiamata "D \(\times\) n" \(= \VelAng_v\cdot d \), quindi  il limite diventa: \(d \cdot \VelAng_v < \text{"D}\times \text{n"}_{lim} \alpha = S \), con \(\alpha\simeq 0.8\) fattore di sicurezza.

\sottosottosezione{Sfere di Acciaio e Ceramica}
In alternativa alle sfere solo di acciaio, esistono viti a ricircolo di sfere in cui sono alternate sfere di acciaio e di ceramica. In particolare le seconde oltre a essere costruite con un volume leggermente inferiore, hanno una massa inferiore, perciò quando vengono messe in movimento le sfere di acciaio quelle di ceramica vanno in contro-rotazione, così facendo il contatto tra sfere vede anch'esso rotazione, mentre permane lo strisciamento sui punti di contatto tra sfere cercamiche e gola. Questa configurazione permette di raggiungere velocità limite della singola sfera \( \DperN_{lim} \) molto superiori, anche doppi.

\begin{figure}[h]
    \centering
    \includegraphics[width=0.12\textwidth]{Immagini/Viti_A_C.png}
    \caption{Viti a ricircolo di sfere in acciaio e ceramica}
\end{figure}

\sottosottosezione{Vincolo 1 su p,d}
A partire da \( d\cdot \VelAng_v \leqslant S = \DperN_{lim} \alpha \), sostituendo la velocità della vite con la velocità del carico traslante, ed esplicitando il passo della vite, si ottiene una relazione di vincolo per passo e diametro della vite:
\[ p \geqslant \frac{\dot{x}_{max} 2\pi }{S} d \]

Nota bene che \(\dot{x}_{max}\) solitamente non è un dato progettuale, è possibile sceglierlo entro limiti termici/strutturali.

\sottosezione{Velocità critica della vite}
La vite è naturalmente sbilanciata, ruotando crea una forzante armonica alla frequenza della velocità di rotazione \( \VelAng_v \), che di valori prossimi alla velocità naturale flessionale \(\omega_{N,flex}\) porta ad avere forti vibrazioni.
Per ridurre gli effetti delle vibrazioni si possono adottare due soluzioni.

\sottosottosezione{Viti smorzate}
Per limitare le vibrazioni è possibile utilizzare viti con un anima in elastomero che vada a smorzare le vibrazioni, inoltre essendo al centro del diametro della vite non andrà a comprometterne la rigidità. Valori tipici di smorzamento di viti smorzate sono \(\epsilon \simeq 0.3 \div 0.5\), contro i \(\epsilon \simeq 0.0x \) di viti normali.

\sottosottosezione{Limitare la velocità di rotazione}
Per evitare di ottenere sulla vite vibrazioni elevate basta evitare di passare per frequenze naturali, cioè \( \VelAng_v < \omega_{N,flex}(x) \), dove \( \omega_{N,flex}(x) \) è dipendente dalla posizione in cui si trova la chiocciola, perché questa solitamente scorre su di una guida in modo da evitare sia la vite a dover sopportare il peso del carico, e quindi va ad aumentare la rigidezza localmente. Il problema risulta complicato.

\paragrafo{Applicazioni velocità in funzione di x:}
Noto il modello elastodinamico della vite, o effettuate opportune misurazioni, è possibile ottenere l'andamento della pulsazione naturale per le varie posizioni della chiocciola \(x\), quindi è possibile variare la  velocità massima della vite in funzione di \(x\). Non semplice.

\paragrafo{Approccio conservativo:}
In alternativa è possibile utilizzare un approccio conservativo in cui si va a limitare la velocità della vite alla minore delle pulsazioni naturali flessionali \( \VelAng_v < \omega_{N,flex}^{min} \). Non è la soluzione ottimale.

Considerando la massa della sola vite\footnote{Carico e madrevite andranno a gravare sulla guida.} la pulsazione naturale flessionale vale:
\[ \omega_{N,flex} = K_v \sqrt{\frac{E \cdot I_f}{M_v \cdot L^3}} \]
con \( K_v \) coefficiente legato al tipo di vincolo; \( E \) modulo di Joung del materiale; \( I_f = \frac{\pi}{64}d^4 \) momento di inerzia; \( M_v \) massa della vite; \( L \) lunghezza della vite.

\sottoparagrafo{Coefficiente di vincolo:}
Il coefficiente di vincolo dipende dal tipo di vincolo presente all'estremo della vite, può essere: libero; appoggio; incastro\footnote{Nota bene non è un incastro di tipo strutturale, è un incastro solo in termini flessionali, non necessariamente andrà a limitare la rotazione della vite.}.
Queste tre possibilità risultano in 4 combinazioni:
\begin{itemize}
    \item Incastro + Libero: \(K_v = 3.5\)
    \item Appoggio + Appoggio: \(K_v = \pi^2\)
    \item Appoggio + Incastro: \(K_v = 15.4\), soluzione preferita.
    \item Incastro + Incastro: \(K_v = 22.4\), questa è la soluzione col coeffiente maggiore. Tuttavia se la vite si riscalda e quindi si dilata, tende a spanciare; inoltre richiede montaggio preciso.
\end{itemize}

\sottoparagrafo{Vincolo 2 su p,d:}
A partire dalla disequazione \( \VelAng_v^{max} \leqslant \omega_{N,flex}^{min} \alpha \), con \( \alpha \simeq 0.6\div 0.8 \) coefficiente di sicurezza che considera come l'ampiezza inizi a salire prima del valore esatto di pulsazione naturale, sostituisco i vari elementi, effettuo qualche manipolazione e si ottiene:
\[ p \geqslant \frac{8\pi L^2 \dot{x}^{max} \sqrt{\rho}}{K_v \alpha \sqrt{E}} \cdot \frac{1}{d} \]
Disequazione per cui risulta evidente la dipendenza inversa tra passo e diametro; quindi che per velocità superiori il vincolo peggiora; che per lunghezze superiori il vincolo peggiora quadraticamente; che per coefficienti di vincoli inferiori il vincolo peggiora.

\begin{figure}[h]
    \centering
    \includegraphics[width=0.4\textwidth]{Immagini/vel_critica_vite_2sol.png}
    \includegraphics[width=0.2\textwidth]{Immagini/vite_guida_MV.png}
\end{figure}

\sottosezione{Vincoli su passo e diametro della vite}
Andando a unire i due vincoli ricavati: \(p \geqslant \frac{\dot{x}_{max} 2\pi }{S} d\) e \( p \geqslant \frac{8\pi L^2 \dot{x}^{max} \sqrt{\rho}}{K_v \alpha \sqrt{E}} \cdot \frac{1}{d} \), la curva ottenuta sarà come in figura. Va evidenziato il forte limite per diametri ridotti.

\sottosottosezione{Esempio}
Fosse richiesto un \( C_D \) "basso", e quindi occorra utilizzare \(d\) "piccoli". In questo caso utilizzare viti acciaio e ceramica non ha senso, perché andrebbe a influenzare unicamente valori di diametro superiori. In questo caso utilizzare vincoli più rigidi potrebbe aiutare, perché porterebbero ad un aumento del numero di punti utilizzabili, tuttavia occorre valutare se i valori aggiunti possano essere utilizzati nell'applicazione in esame o meno.

\begin{figure}[h]
    \centering
    \includegraphics[width=0.6\textwidth]{Immagini/viti_ricir_vincoli_p_d.png}
    \includegraphics[width=0.2\textwidth]{Immagini/viti_ricir_esempio_vincoli.png}
    \caption{1-3 relative a vincolo e dipendenze, 4 relativa all'esempio}
\end{figure}

\sottosezione{Rigidezza Assiale}
Il moto assiale del carico porta a vibrazioni nel sistema, che possono causare problemi di precisione e accuratezza nel moto del carico.
Un sistema reale ha infiniti modi di vibrare, tuttavia la conoscenza dei parametri è tale per cui una semplificazione con un modello a 2gdl è comparabile con risultati con modelli a più gdl.

\sottosottosezione{Modello a 2gdl}
Concentro tutte l'elasticità nel punto di traslazione della chiocciola, mentre considero il resto del sistema come rigido, occorre valutare una \(K_{eq}\) e \( M_{eq} \), in cui sono considerati i vari contributi:
\[ K_{eq} = \left[ \frac{1}{K_\text{ass,vite}} + \frac{1}{K_\text{tors,vite}}\left(\frac{p}{2\pi}\right)^2 + \frac{1}{K_\text{ass,chiocciola}} + \frac{1}{K_\text{ass,cuscinetti}} + \frac{1}{K_\text{ass,giunto}} \right]^{-1} \] \[ \simeq \left[ \frac{1}{K_\text{ass,vite}} + \frac{1}{K_\text{tors,vite}}\left(\frac{p}{2\pi}\right)^2 \right]^{-1} \]
Dove i contributi più rilevanti sono tendenzialmente i primi due, che sono dipendenti da p, d quindi non necessitano del catalogo.

\sottosottosezione{Rigidezza Torsionale}
La rigidezza torsionale è data da \(K_\text{torsionale} = \frac{G I_p}{L_\text{vite}} \) con \(I_p=\frac{\pi}{32} d^4\) momento di inerzia polare, G modulo elastico tangenziale.
Non dipende molto dalla chiocciola, perciò si considera come costante al variare della posizione.

\sottosottosezione{Rigidezza Assiale}
La rigidezza assiale è data da \( K_\text{assiale} = \frac{EA}{L_\text{assiale}} \), con \( A \) sezione della vite.
In questo caso l'effetto della chiocciola NON è trascurabile.
In particolare a seconda del tipo di vincolo sulla vite cambia sensibilmente.

\paragrafo{Caso unico vincolo assiale:}
Se sulla vite è posto un unico vincolo assiale (AA,AI,IL), allora un estremo è libero di muoversi assialmente, perciò non sarà interessato dalla rigidezza assiale.
In questo caso la lunghezza assiale è determinata dalla distanza tra motore e chiocciola.
In termini di rigidezza minima questa si verifica per lunghezza assiale massima, ossia quando la chiocciola è posta in prossimità dell'estremo assialmente libero.

\paragrafo{Caso doppio vincolo assiale:}
Se sulla vite è posto un doppio vincolo assiale (II), allora entrambi gli estremi saranno interessati dalla rigidezza assiale.
In questo caso la vite viene divisa in due molle equivalenti poste in parallelo, di lunghezze \(x\) (posizione della chiocciola), \(L_v - x\) (resto della vite)\footnote{In realtà la lunghezza assiale considera non la lunghezza totale della vite ma da vincolo a vincolo, tuttavia è spesso una buona approssimazione.}: \(K_\text{ass,vite}(x) \simeq \frac{EA}{x} + \frac{EA}{L_v - x} \).
La rigidezza ha massimo in corrispondenza di \( K_\text{ass,vite}^\text{max} (x=0,L_v) \rightarrow \infty \), nota bene che se questa rigidezza è infinita, nel calcolo della rigidezza equivalente per il modello a 2gdl i parametri prima ritenuti trascurabili non potranno più esserlo.
La rigidezza ha minimo in corrispondenza di \(K_\text{ass,vite}^\text{min} = \frac{4EA}{L_v} \).

\begin{figure}[h]
    \centering
    \includegraphics[width=0.6\textwidth]{Immagini/rigidezza_ass_1vs2_vincoli.png}
    \caption{Rigidezza per vincoli assiali: unico a sx, doppio a dx}
\end{figure}

\paragrafo{Usare rigidezza equivalente:}
Il valore di \(K_eq\) di per sè non crea problemi, tuttavia considerando il modello a 2 gdl e considerando il motore incastrato, si otterrà una pulsazione di antirisonanza \(\omega_z = \sqrt{\frac{K_{eq}}{M_{eq}}}\) relativa alla funzione di trasferimento data da \( \frac{\Theta(s)}{C_m(s)} \).
La massa equivalente è data da \( M_{eq}(d) \simeq M_\text{carico} + M_\text{chiocciola} + \frac{M_\text{vite}}{3} \); la rigidezza è funzione di \( K_{eq}(p,d,x) \), perciò \(\omega_z (p,d,x)\), e la pulsazione di antirisonanza ha implicazioni sulla banda passante \( \omega_{bv} \leqslant \omega_z \cdot [50\% \div 70\%] \), quindi su inseguimento del riferimento.

\paragrafo{Rappresentazioni in grafico p,d:}
In termini di grafico passo, diametro: per d piccoli, \( K_{ass} \) è minore quindi più critico; per p grandi diventa più rilevante l'effetto torsionale perchè una torsione per passo grande implica un grande spostamento assiale.

\sottosezione{Verifica Statica}
Per verificare staticamente una vite si utilizzano due verifiche:
\begin{itemize}
    \item Deformazione statica della vite, legata a forza peso;
    \item Instabilità al carico di punta, ossia deformazione legata a forza di compressione sulla vite, di tipo euleriano.
\end{itemize}
Tuttavia le verifiche statiche sono meno severe delle verifiche dinamiche, il vincolo che si può trovare in termini di diametro minimo esclude una area di grafico già esclusa dalla velocità critica della vite.

\begin{figure}[h]
    \centering
    \includegraphics[width=0.5\textwidth]{Immagini/vincolo_antirisonanza.png}
    \caption{Rappresentazione vincolo p,d per antirisonanza sx, grafico di tutti i vincoli dx}
\end{figure}

\sottosezione{Gioco Assiale} \label{giocoViti}
Prima ancora della scelta di p, d è opportuno scegliere il gioco, che tipicamente è indipendente da p,d perciò nei cataloghi è in sezione a sè stante.
Il problema del gioco è legato alla grandezza delle gole rispetto le sfere, questo comporta forze che non sono esattamente assiali, ma hanno componente normale.

\paragrafo{Soluzione 1:}
Utilizzo di sfere maggiorate, che vadano a occupare più spazio all'interno delle gole, tuttavia questo comporta un aumento dell'attrito e maggior richiesta di manutenzione.

\paragrafo{Soluzione 2:}
Utilizzo di Precarico o Shift di passo, in questo caso va introdotto un certo \(\Delta\) in un passo, di modo che le forze normali si annullino. Per poter utilizzare un precarico occorre utilizzare una chiocciola apposita che, tramite una molla regolabile, forza il contatto delle sfere con le gole. Questo aumenta di conseguenza l'attrito, è sempre applicata, e viene fissata non oltre il \(10\%\) di \(C_D\), il cui calcolo va modificato di conseguenza: \( C_D \geqslant \frac{f_s F_a}{0.9} \sqrt[3]{\frac{L_N}{10^6}} \).

In alcuni casi il sistema è precaricato per configurazione, per esempio quando la vite è posta verticalmente.

\begin{figure}[h]
    \centering
    \includegraphics[width=0.4\textwidth]{Immagini/gioco_viti_ricircolo.png}
    \caption{Gioco nelle viti fig 1,2; precarico fig 3,4}
\end{figure}

\sezione{Passo ottimo}
Le stesse valutazioni fatte per i riduttori (vedi \ref{tau_ottimo}, a pagina \pageref{tau_ottimo}) si possono trasporre alle viti a ricircolo di sfere, per cui si parlerà di \(p_{opt}=2\pi \sqrt{\frac{(J_m+J_r)\acc{x}^{RMS}}{(F_a/\eta_r)^{RMS}}}\).

\paragrafo{Effetto del Diametro:}
Il diametro entra nel calcolo dell'inerzia, in particolare, sostituendo la massa \(J_v=\frac{1}{2}L\rho \pi \frac{d^4}{16} = \alpha d^4\), perciò \(p_{opt}=2\pi \sqrt{\frac{(J_m+\alpha d^4)\acc{x}^{RMS}}{(F_a/\eta_r)^{RMS}}}\), ossia il passo della vite varia circa quadraticamente con il diametro.
\begin{figure}[h]
    \centering
    \includegraphics[width=0.3\textwidth]{Immagini/passo_ottimo_libero.png}
    \caption{Passo Ottimo Libero e Vincoli}
\end{figure}

\paragrafo{Stima coppia del motore per diversi passi e diametri:}
Quando dovesse essere necessario fare valutazioni di varie tipologie di viti aventi diverse combinazioni di passi e diametri, per evitare di fare tutti i conti ogni volta, è possibile sfruttare la disuguaglianza triangolare per cui \(RMS(a+b) \leqslant RMS(a) + RMS(b) \). Nel caso in esame, al variare del diametro cambia \(J_v\), mentre col passo varia \(\tau_v\), per cui la disuguaglianza triangolare risulta essere:
\[ C_m^{RMS} \leqslant (J_m + J_v) \frac{\dots{x}_{RMS}}{\tau_v} + \tau_v \frac{F_A^{RMS}}{\eta} \]