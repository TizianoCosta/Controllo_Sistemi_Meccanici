\intro
Nel seguente file sono presenti gli appunti del corso di Controllo di Sistemi Meccanici relativi all'anno accademico 2024-2025.

\vspace{5pt}
\hline
\vspace{5pt}

Volevo innanzitutto fare un ringraziamento speciale al mio editor per il prezioso aiuto a scovare refusi e errori nelle formule, e un ringraziamento a tutti i ragazzi del corso LM IMC per gli \(n\rightarrow \infty\) momenti di supporto e aiuto senza i quali avrei capito meno di metà di quanto qui esposto.

\vspace{5pt}
\hline
\vspace{5pt}

L'esame è composto da esercizi e teoria, tuttavia, a differenza di altri corsi, sono pressoché interscambiabili.
La parte di esercizi si compone di diversi punti, molti dei quali domande tipiche, alcune sono invece sottigliezze o aspetti apparentemente di secondaria importanza. Conta molto giustificare le scelte effettuate.
Le formule utili per gli esercizi che possono essere ricavate, per esempio quelle dei \(C_V,C_A,\tau_{opt},\dots\) conviene impararle a memoria, il tempo a disposizione non è sufficiente per poterle ricavare.
La parte di teoria dovrebbe essere più tranquilla, come verifica delle conoscenze in cui è fattibile aumentare il voto rispetto quanto preso nella parte di esercizi.
Nota bene: pare che al primo tentativo che si effettua l'esame segni il nome e penalizzi successivi tentativi; cercare di fare buona la prima.

\vspace{5pt}

I miei migliori auguri di buon esame.