\intro
Nel seguente file sono presenti gli appunti del corso di Controllo di Sistemi Meccanici relativi all'anno accademico 2024-2025.

\bigskip
\rule{\linewidth}{0.4pt}
\bigskip

Volevo innanzitutto fare un ringraziamento speciale al mio editor per il prezioso aiuto a scovare refusi e errori nelle formule, e un ringraziamento a tutti i ragazzi del corso LM IMC per gli \(n\rightarrow \infty\) momenti di supporto e aiuto senza i quali avrei capito meno di metà di quanto qui esposto.

\bigskip
\rule{\linewidth}{0.4pt}
\bigskip

L'esame è composto da esercizi e teoria, tuttavia, a differenza di altri corsi, sono pressoché interscambiabili.
La parte di esercizi si compone di diversi punti, molti dei quali domande tipiche, alcune sono invece sottigliezze o aspetti apparentemente di secondaria importanza. Conta molto giustificare le scelte effettuate.
Le formule utili per gli esercizi che possono essere ricavate, per esempio quelle dei \(C_V,C_A,\tau_{opt},\dots\) conviene impararle a memoria, il tempo a disposizione non è sufficiente per poterle ricavare.
La parte di teoria dovrebbe essere più tranquilla, come verifica delle conoscenze in cui è fattibile aumentare il voto rispetto quanto preso nella parte di esercizi.

Ogni esame dura 3 ore, gli esercizi sono a punteggio, quindi cerca di fare rapporto punteggio su 32 per capire quanto tempo impiegare per ciascun esercizio. I primi punti sono i più facili (di solito) e valgono di più, non conviene lasciare un esercizio completamente in bianco.

Esame 1:
\begin{itemize}
    \item Pianificazione del moto: Continuità fino al jerk; primo tratto da 0 a punto con velocità costante non nulla (polinomiale grado 7); terzo tratto con passaggio per punti.
    \item Controllo: Ricavare i parametri del sistema dato grafico; sintonizzazione del controllo;
    \item Dimensionamento: Vite a ricircolo di sfere su piano inclinato; moto con legge triangolare in velocità simmetrica; scelta del motore obbligata dal calcolo della pulsazione naturale flessionale della vite (occorreva sapere a memoria come ricavarla, incluso il modulo di Joung dell'acciaio, i coefficienti di rigidezza per appoggio incastro e incastro incastro, ecc)
\end{itemize}

Orale 1:
\begin{enumerate}
    \item Domande su qualcosa sbagliato nello scritto
    \item Motore collegato a vite a ricircolo di sfere con un carico traslante che vibra, come risolvere? (domanda per il 30)
    \item Acceleration Feedback, Load observer
    \item Variazione di fase al variare di \(\rho\)
    \item Luogo delle radici di controllo di posizione non colocato
    \item Come cambia il luogo delle radici di controllo di posizione colocato al variare di \(K_{pv}\) e come cambia \(\omega_{bv}\)
\end{enumerate}

Esame 2:
\begin{itemize}
    \item Pianificazione del moto: Legge tra punto e punto con velocità di partenza e arrivo non nulle, con continuità di accelerazione (polinomiale di grado 3), da disegnare con buona precisione;
    \item Controllo: Dato grafico di funzione di trasferimento di anello di corrente e Gvm, ricavare i parametri, sintesi controllore, valutazioni su filtri;
    \item Dimensionamento: Motore, riduttore, ruote coniche, carico; legge di moto trapezoidale in velocità non simmetrica con periodo di sosta; dinamica; dimensionamento riduttore; calcolo dell'errore statico di posizionamento del carico noti i vari giochi e la coppia statica; ricavare antirisonanza e risonanza del sistema.
\end{itemize}

C'è una regola per cui dalla prima volta in cui provi l'esame hai l'anno accademico corrente per provare l'esame con punti fino a 32, oltre l'anno accademico il conteggio punti arriva a 30.
Non portare a casa la bruttacopia, anche se è solo un foglio bianco.

\vspace{5pt}

I miei migliori auguri di buon esame.