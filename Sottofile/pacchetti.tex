\usepackage[utf8]{inputenc}

%da usare per subfile
\usepackage[subpreambles=true]{standalone}
%se ho subfile tex dentro cartelle
%\usepackage{import} 

\usepackage{color}

%American Mathematic Society Symbols, sono simboli matematici particolari extra
\usepackage{amssymb}
\usepackage{amsmath}   % <-- for \eqref e \parallel

% Per inserire elenchi numerati con numeri romani
\usepackage{enumitem}

%Per mettere link dentro i file
%Per permettere una table of content cliccabile
\usepackage[hidelinks]{hyperref}

\hypersetup{
    colorlinks=false, %set true if you want colored links
    linktoc=all,     %set to all if you want both sections and subsections linked
    linkcolor=black,  %choose some color if you want links to stand out
}

%Simboli molto particolari tipo il pallino o note musicali
%\usepackage{textcomp}

%\usepackage{svg}
\usepackage{graphicx}

\usepackage{mathtools}

% Estensione per le funzioni array e tabular
%\usepackage{array}

\usepackage{circuitikz}
\usepackage{tikz}
\usetikzlibrary{shapes.geometric, positioning}
%\usepackage{wrapfig} 

\graphicspath{{./Immagini/}} %percorso immagini


%Subfile per funzionare bene va caricato alla fine di tutto
%\usepackage{subfiles}
