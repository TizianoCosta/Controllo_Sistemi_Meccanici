\capitolo{Controllo del Moto per Trasmissione Rigida Senza Gioco}
Nell'industria i controllori sono di tipo P, PI, quindi le valutazioni saranno su questi.
Il modello di riferimento è quello classico del sistema motore, riduttore, carico a cui si aggiunge l'espressione base del motore:
\[
    C_m(t) = J \AccAng(t) + f \VelAng(t) + C_e (t) \\
    C_m(t) = K_T i(t)
\]

Che trasformata in Laplace diventa:
\[
\begin{cases}
    C_m(s)=(Js^2+fs)\theta(s) + C_e (s) \\
    C_m(s)=K_T I(s)
\end{cases}
\]

In cui sono state raggruppate tutte le varie inerzie, coefficienti di attrito e coppie legate a forze esterne.
Il sistema è a 1gdl.

\paragrafo{Schema Multi-Anello:}
In un sistema motore, riduttore, carico ci sono solitamente diversi anelli di controllo. Partendo dall'interno verso l'esterno ci sono gli anelli di: (tensione)\footnote{Nominato, ma non viene esaminato.} corrente, velocità e posizione.

\sottoparagrafo{Anello di corrente:}
L'anello di corrente è indipendente dalla meccanica; dipende dall'elettrica, dall'elettronica e dal controllo. Ha anche bande passanti caratteristiche ad alta frequenza \([kHz]\).

\sottoparagrafo{Anello di posizione:}
L'anello di posizione dipende: anello di velocità; trasduttore di posizione; controllo di posizione. In applicazioni tipiche ha banda passante inferiore i \(50 Hz\). 

\paragrafo{Feedback vs Feedforward:}
Nella trattazione classica dei controlli si tende a valutare soprattutto il sistema con feedback (catena chiusa), però tende a limitare le prestazioni. Un approccio più flessibile utilizza un feedback ben sintonizzato e un feedforward (catena aperta) basato sulla conoscenza (approssimata) del sistema fisico e del modello, che si traduce in una stima (nota a priori) di corrente\footnote{Nel caso dell'anello di velocità.} richiesta per la movimentazione.

\sezione{Anello di Velocità}
\import{Immagini/}{modello_complessivo}

L'anello di velocità dipende da: anello di corrente; meccanica; trasduttori di velocità; controllore di velocità.
Tipicamente ha bande passanti sul centinaio di \([Hz]\).

\sottosezione{Modello Semplificato 1}
Lo schematico in figura è quello completo, tuttavia in prima analisi conviene fare delle approssimazioni:
\begin{itemize}
    \item Coppie esterne vengono trascurate (valido per sovrapposizione degli effetti)
    \item Non utilizzo feedforward
    \item Trasduttore ideale, la misura non ha ritardi o errori, è come se il blocco relativo fosse 1
    \item Anello di corrente ideale, la corrente erogata, è come se il blocco relativo fosse 1
\end{itemize}

\import{Immagini/}{modello_semplificato_1}

\sottosottosezione{Controllo proporzionale}
Considerando il modello semplificato e l'utilizzo di un controllo di tipo proporzionale, ottengo che la funzione di trasferimento del sistema a catena chiusa è 
\[W(s)=\frac{K_{pv}K_T}{K_{pv}K_T+f}\frac{1}{1+s\frac{J}{f+K_{pv}+K_T}}\]
dove \(\tau = \frac{J}{f+K_{pv}+K_T}\) è la costante di tempo\footnote{Nota bene: si può parlare di costanti di tempo solo per sistemi del primo ordine.}.

\paragrafo{Banda passante:}
In un sistema del primo ordine la banda passante è data da \(\omega_B=\frac{1}{\tau}\), tuttavia considerando trascurabile l'attrito risulta \(\omega_B\simeq \frac{K_{pv}K_T}{J}\). Quest'ultima relazione permette facilmente di sintonizzare il controllore invertendo \(K_{pv}=\frac{\omega_B J}{K_T}\).

\paragrafo{Errore a regime:}
Per un sistema del primo ordine il guadagno in continua/regime è dato da \(\frac{K_{pv}K_T}{K_{pv}K_T+f}<1\), che è necessariamente minore di 1, ossia l'errore a regime non è nullo. Questo è molto negativo. Non potendo impostare un guadagno proporzionale infinito, occorre necessariamente valutare un controllore di altro tipo che permetta di ottenere un errore a regime nullo, di modo da avere un guadagno in continua unitario.

\sottosottosezione{Controllo Proporzionale Integrale}
Un controllore PI è del tipo \(C_V(s)=K_{pv}\frac{1+sT_{iv}}{sT_{iv}}\), per cui il sistema a catena chiusa diventa 
\[W_v(s)=\frac{K_{pv}K_T(1+sT_{iv})}{s^2T_{iv}J + sT_{iv}(f+K_{pv}K_T)+K_{pv}K_T}\]
Come era voluto, il sistema così ottenuto ha guadagno in continua unitario.
Però aumenta il numero di poli, diventa di secondo ordine\footnote{Fare riferimento a quanto presente in appendice \ref{sistemi_ordine_2}} (cosa che deve mettere in allarme per una serie di implicazione) e viene aggiunto uno zero reale (che ha anch'esso forti implicazioni). Tuttavia il grado relativo rimane 1.

\paragrafo{Effetto dello zero reale:}
Per valutare bene l'effetto dello zero nel sistema a catena chiusa adotto una parametrizzazione:
\[\frac{1+\frac{s}{\alpha \xi \omega_n}}{\left(\frac{s}{\omega_n}\right)^2+\frac{2\xi}{\omega_n}s+1}\]
Utilizzare \(\alpha\) permette di valutare dove si trova lo zero rispetto i poli del sistema
\[
\begin{cases}
\alpha >> 1 \text{ \ zero in alta frequenza} \\
\alpha \simeq 1 \text{ \ zero vicino ai poli} \\
\alpha << 1 \text{ \ zero in bassa frequenza}
\end{cases}
\]
In particolare si nota come la presenza dello zero vada a incrementare la sovraelongazione.

\textbf{La sovraelongazione dipende da \(\xi\) e dalla posizione dello zero.}

\begin{figure}[h]
    \centering
    \includegraphics[width=0.45\textwidth]{Immagini/risposta_gradino_ord1_tempo.png}
    \includegraphics[width=0.45\textwidth]{Immagini/influenza_zero_su_sovraelong.png}
    \caption{Risposta al gradino di sistema di ordine 1 (sx); \(M_p(\alpha)\) in risposta al gradino (dx)}
\end{figure}

\sottoparagrafo{Scomposizione sovraelongazione poli e zero:}
A partire dal sistema a catena chiusa, vado a scomporre la componente legata al sistema del secondo ordine dal resto, cioè un sistema del secondo ordine derivato \(H(s)=\frac{1}{\left(\frac{s}{\omega_n}\right)^2+\frac{2\xi}{\omega_n}s+1} + \frac{1}{\alpha \xi \omega_n}\frac{s}{\left(\frac{s}{\omega_n}\right)^2+\frac{2\xi}{\omega_n}s+1}\), che nel tempo diventa \(h(t)=h_{2 ord}(t) + \frac{1}{\alpha \xi \omega_n} \derivata{h_{2 ord}(t)}{t}\).

Nella risposta al gradino avrò una componente legata alla risposta al gradino del sistema del secondo ordine e una componente legata alla risposta all'impulso\footnote{Impulso che è la derivata del gradino.} del sistema del secondo ordine.
\begin{itemize}
    \item Se cala lo smorzamento, a parità del resto, la sovraelongazione aumenta per effetto dello zero: \(\downarrow \xi \ \uparrow M_p\)
    \item Se \(\alpha >>1\) l'effetto legato allo zero diventa trascurabile
    \item Se \(\alpha <<1\) l'effetto legato allo zero è tanto più severo tanto è minore lo smorzamento
\end{itemize}

Quindi per attenuare la sovraelongazione conviene:
\begin{itemize}
    \item Posizionare gli zeri ad alta frequenza \(>> \xi \omega_n\)
    \item Se \(\xi << 1\) aumenta la sensibilità allo zero quindi lo zero va portato ancora  a più alta frequenza
    \item Se non è possibile intervenire su \(\alpha, \xi\), occorre intervenire sulla legge di moto
\end{itemize}

\paragrafo{Legge di Moto:}
A leggi di moto più dolci si associano sovraelongazioni legate allo zero più trascurabili. A leggi di moto non dolci si associano sovraelongazioni legate allo zero maggiori.