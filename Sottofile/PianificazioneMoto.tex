\capitolo{Pianificazione del Moto}
Pianificazione del moto significa scegliere la legge di moto, ossia \textit{La relazione matematica che esprime il moto in termini di posizione, velocità e accelerazione di un asse, motore o carico, in funzione di un parametro indipendente.}
\begin{itemize}
    \item Tempo, \(t [s]\)
    \item Posizione di un altro asse, \(q [m]\)
    \begin{itemize}
        \item Virtuale: camme elettroniche come clock di macchine
        \item Reale
        \begin{itemize}
            \item 2 assi meccanicamente accoppiati: camme meccaniche
            \item 2 assi accoppiati solo con controllo: camme elettriche
        \end{itemize}
    \end{itemize}
\end{itemize}

\sezione{Scelta delle Leggi di Moto}
Le specifiche in termini di legge di moto sono tendenzialmente poche, legate a: tempo di ciclo, spazio percorso, velocità minima, massima; a partire dalle quali va fatta una ottimizzazione consci di non poter ottenere una legge di moto perfetta.

\sottosezione{Criteri di Scelta}
La scelta del tipo di legge di moto viene effettuata a partire da alcune caratteristiche progettuali:
\begin{enumerate}
    \item Minimizzazione di velocità massima o RMS
    \item Minimizzazione di accelerazione massima o RMS
    \item Minimizzazione dell'energia
    \item Minimizzazione della potenza
    \item Garantire realizzabilità della legge di moto in termini dinamici\footnote{In termini di banda passante di motore, controllo, di spettro della legge di moto e vibrazioni.}
\end{enumerate}