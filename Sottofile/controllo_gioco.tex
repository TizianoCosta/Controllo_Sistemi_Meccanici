\capitolo{Controllo del Moto in presenza di Gioco}
Il gioco è caratterizzato da una variazione dell'inerzia equivalente legata a cambio di segno di \(T_2\) (questo non si verifica in ogni situazione, vedi {\color{red} Riferimento quando nelle viti a ricircolo si parlò di gioco}). 
Motore e carico possono essere accoppiati per fianco destro o sinistro perciò \(J_{eq} = J_m + \tau^2 J_c\); motore e carico possono essere disaccoppiati \(J_{eq} = J_m\).

\sezione{Problema del gioco}
Il problema legato alla variazione di inerzia è il seguente:
\[
\begin{cases}
    \text{Accoppiati: } J_{eq} = J_m (1+\rho) \ \rightarrow \ \omega_{bv}^{min} \simeq \frac{K_{pv}K_T}{J_m(1+\rho)} \\
    \text{Disaccoppiati: } J_{eq} = J_m \ \rightarrow \ \omega_{bv}^{max} \simeq \frac{K_{pv} K_T}{J_m}
\end{cases}
\]
Ossia passando da carico e motore accoppiati a disaccoppiati, per \(K_{pv}\) costante, la banda passante varia di un fattore \(1+\rho\), l'aumento di banda passante è associato a avvicinamento a \(\omega_I,\omega_{tv}\), che potrebbe portare a calo di margine di fase, e instabilità.

Il gioco è caratterizzato da passaggio imprevedibile da motore carico accoppiati e disaccoppiati, e il cambio tende ad autoalimentarsi, si instaura un ciclo limite di oscillazione non armonica permanente.
Questo problema affligge controllo colocato e non colocato.

\paragrafo{Guadagno di velocità:}
Per limitare gli effetti legati al gioco occorre fare \textbf{riduzione} di \(\mathbf{K_{pv}}\), che va dimensionato in modo da ottenere \(\omega_{bv}^{max} < \omega_{bv}^{lim}\) o comunque tale da garantire un certo margine di fase\footnote{La scelta del margine di fase non dovrebbe essere eccessivamente conservativa, \(m_\phi = 20^\circ\), serve per evitare l'autoalimentazione, dovrebbe servire un numero ridotto di volte per ciclo.} anche per c-m disaccoppiati.

\sottosezione{Rapporto di inerzia}
Il problema del gioco viene semplificato quando viene effettuata la \textbf{riduzione del rapporto di inerzia}, perché porta ad una minore differenza di inerzia tra c-m disaccoppiati/accoppiati, quindi ad avere bande passanti e margini di fase simili.
Tuttavia la scelta di \(\rho\) non dovrebbe portare a utilizzo di riduttori con alto rapporto di trasmissione e gioco elevato.

\begin{figure}[h]
    \centering
    \includegraphics[width=0.45\textwidth]{Immagini/gioco_banda_passante_rho.png}
    \caption{Effetto del gioco: \(\omega_{bv}(\rho)\)}
\end{figure}

%%%%%
{{\color{red}}Perchè la curva grigio scuro con H alta ha quella forma, ho scritto "legata al gioco per garantire omega_bv^max < limite", ma che significa?}

\paragrafo{Effetto del rumore:}
Nel grafico le curve calano per \(\rho\) elevati perché aumenta il \(K_{pv}\) associato, che è quello che fa aumentare gli effetti del rumore. Tuttavia non c'è una relazione diretta tra \(\rho\) e aumento del rumore, la causa è l'inerzia totale \(J\), infatti, il calcolo del guadagno per l'anello di posizione è \(K_{pv} = \frac{J \omega_{bv}^{des}}{K_T}\), dove facilmente ci si rende conto di questa relazione. Abbassare \(\rho\) aumentando l'inerzia potrebbe peggiorare gli effetti del rumore.

\paragrafo{Rapporto di trasmissione ottimo:}
Riprendendo il rapporto di trasmissione ottimo come visto in {\color{red} Aggiungere riferimento, inserire formula} si ritrova, utilizzando una logica diversa, un valore che vale anche per il controllo, dato che porta ad avere \(\rho\) ridotto {\color{red}Come si collega il rapporto di trasmissione ottimo con il rapporto di inerzia, e come mai tao opt porta ad avere rho minore?}

\sottosezione{Acceleration Feedback}

%% inserire schematico pag 86

Acceleration Feedback è una tecnica che consiste nel retroazionare l'accelerazione e portarla al motore come corrente, così facendo la relazione tra accelerazione e corrente diventa \(\frac{s^2\theta}{i} = \frac{K_T}{J + K_TK_A}\), ossia viene aumentata virtualmente l'inerzia lato motore di una quantità \(K_TK_A\), da cui si ottiene un rapporto di inerzia virtuale minore, quindi meno sensibile al gioco, \(\rho^\text{virtuale}=\frac{\tau^2 J_c}{J_m + K_TK_A}\).

Tuttavia la misura di accelerazione non è possibile. L'accelerazione può essere stimata a partire dalla misura di posizione utilizzando una doppia derivata e opportuno filtro (il filtro è fondamentale perché la derivata va a amplificare il rumore) oppure utilizzando stimatori complessi (Kalman). In entrambi i casi occorre una misura della posizione molto precisa, inoltre i risultati comunque non sono particolarmente efficaci.\label{misura_acc}

\sottosezione{Load Observer}

%% inserire schematico pag 87

Load Observer è una tecnica che considera il sistema ideale composto dal solo motore, mentre le non idealità vengono consensate in un contributo di coppia di disturbo (gioco e elasticità della trasmissione e attrito).
Se questo contributo di disturbo fosse determinabile, potrebbe essere cancellato (in feedback o eventualmente in feedforward), il sistema ottenuto sarebbe quello ideale solo motore, il carico non viene visto.

Non vedere il carico è il maggior difetto, rimanendo in catena aperta è fondamentale effettuare una opportuna scelta della legge di moto.
Tuttavia significa anche che \(\rho_\text{virtuale} \rightarrow 0\), quindi non c'è sensibilità a gioco, risonanza e antirisonanza virtuali si elidono quindi non c'è sensibilità all'elasticità.

Anche in questo caso è fondamentale la misura di accelerazione, che però ha tutta una serie di problematiche (vedi \ref{misura_acc}).

\paragrafo{Stima della corrente:}
In ingresso al motore viene aggiunta una corrente \(i_{LO}\) tale da generare una coppia uguale e opposta alla coppia di disturbo, in modo si elidano.
\[\begin{cases}
    C_m = K_T i \\
    C_m = J_m \AccAng_m + C_\text{disturbo}
\end{cases} \ \rightarrow \ C_\text{disturbo} = C_m - J_m \AccAng_m = K_T i - J_m \AccAng_m \rightarrow i_{LO} = \frac{K_T i - J_m \AccAng_m}{K_T} \]  

\sottosezione{Esercizio}

\sezione{Autotuning}
Con Autotuning si intendono le tecniche di sintonizzazione automatica del controllo a partire da misure sperimentali.
Step:
\begin{enumerate}
    \item Misura di \(L_v(j\omega)\), utilizzata per cercare \(K_{pv}\) che soddisfi \(m_\phi\) e \(m_a\) desiderati o minimi
    \item Calcolo dell'integrale \(\frac{1}{T_{iv}}\) (noi l'abbiamo visto con metodi empirici)
    \item Misura di \(W_v(j\omega)\)
    \item Verifica dell'utente che potrebbe utilizzare dei filtri per cercare eventualmente di migliorare quanto ottenuto
    \item Misura di \(L_p(j\omega)\), utilizzata per sintonizzare \(K_{pp}\)
    \item Misura di \(W_p(j\omega)\)
\end{enumerate}

Occorre prestare attenzione che l'autotuning si basa sulle misure, questo significa che occorre prestare attenzione a una serie di aspetti: scelta di un opportuno numero di misure, encoder sufficientemente risoluto, applicazione di eccitazione di giusta intensità e durata, e come questi devono essere adattati considerando la presenza di gioco. 